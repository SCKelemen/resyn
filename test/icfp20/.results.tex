\multirow{7}{*}{\parbox{1cm}{\vspace{-0.85\baselineskip}\center{Quadratic}}} & in-order pairs & 52 & 0.41 & 0.25 \\
 & list reverse & 32 & 0.31 & 0.23 \\
 & remove duplicates & 41 & 0.34 & 0.22 \\
 & insertion sort (coarse) & 47 & 0.44 & 0.23 \\
 & selection sort & 72 & 0.47 & 0.23 \\
 & quadratic mergesort & 97 & 0.86 & 0.25 \\
 & quadratic quicksort & 78 & 0.74 & 0.24 \\
\hline\multirow{2}{*}{\parbox{1cm}{\vspace{-0.85\baselineskip}\center{Non-Polynomial}}} & subset sum problem & 28 & 0.31 & 0.22 \\
 & merge sort flatten & 62 & 0.71 & 0.23 \\
\hline\multirow{3}{*}{\parbox{1cm}{\vspace{-0.85\baselineskip}\center{Value-Dependent}}} & insertion sort (fine) & 47 & 4.22 & 0.23 \\
 & BT member & 23 & 1.04 & 0.21 \\
 & BT insert & 29 & 3.21 & 0.23 \\
\hline